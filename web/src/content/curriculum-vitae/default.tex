\documentclass[a4paper]{article}

\usepackage[T1]{fontenc}
\usepackage[utf8]{inputenc}
\usepackage{amsmath}
\usepackage{amssymb}
\usepackage{fancyhdr}
\usepackage{fontspec}
\usepackage{fullpage}
\usepackage{hyperref}
\usepackage{multicol}
\usepackage{textcomp}
\usepackage{xcolor}

\hypersetup{%
  colorlinks=false,
  linkbordercolor={0 0 0},
  pdfborder={0 0 0},
  pdftitle={David Wood's Curriculum Vitae},
  pdfauthor={David Wood},
  pdfsubject={},
  pdfkeywords={}
}

\setmainfont{austeratext}[
  Extension=.otf,
  Path=fonts/,
  UprightFont=*-regular,
  ItalicFont=*-regular-italic,
  BoldFont=*-bold,
  BoldItalicFont=*-bold-italic,
]

\textheight=10in
\pagestyle{fancy}
\raggedright

\def\bull{\vrule height 0.8ex width .7ex depth -.1ex }

\newcommand{\area} [2] {
  \vspace*{-9pt}
  \begin{verse}
    \textbf{#1}   #2
  \end{verse}
}

\newcommand{\lineunder} {
  \vspace*{-8pt} \\
  \hspace*{-18pt} \hrulefill \\
}

\newcommand{\header} [1] {
  {\hspace*{-18pt}\vspace*{6pt} \textsc{#1}}
  \vspace*{-6pt} \lineunder
}

\newcommand{\employer} [3] {
  { \textbf{#1} (#2)\\ \underline{\textbf{\emph{#3}}}\\  }
}

\newcommand{\contact} [3] {
  \vspace*{-10pt}
  \begin{center}
      {\Huge \scshape {#1}} \\
      #2 \\ #3
  \end{center}
  \vspace*{-8pt}
}

\newenvironment{achievements}{
  \begin{list}
    {$\bullet$}{\topsep 0pt \itemsep -2pt}}{\vspace*{4pt}
  \end{list}
}

\newcommand{\schoolwithcourses} [4] {
  \textbf{#1} #2 $\bullet$ #3 \\
  #4 \\
  \vspace*{5pt}
}

\newcommand{\school} [4] {
  \textbf{#1} #2 $\bullet$ #3 \\
  #4 \\
}

\begin{document}

\fancyhead{}
\renewcommand{\headrulewidth}{0pt}

\fancyfoot{}
\fancyfoot[L]{\scriptsize\scshape\color{gray} Generated \today}
\fancyfoot[C]{\scriptsize\scshape\color{gray} David Wood's Curriculum Vitae}
\fancyfoot[R]{\scriptsize\scshape\color{gray} \thepage}

\vspace*{-40pt}

\vspace*{-10pt}
\begin{center}
  {\Huge \scshape {David Wood}} \\
  \vspace{1mm}
  Senior Software Engineer $\cdot$
  Edinburgh, Scotland $\cdot$
  \href{mailto:hello@davidtw.co}{hello@davidtw.co} $\cdot$
  \href{https://davidtw.co}{davidtw.co} $\cdot$
  \href{https://github.com/davidtwco}{@davidtwco} \\
\end{center}

\header{Open Source}
\vspace{1mm}

\textbf{Rust Programming Language} \\
\textit{Compiler Team Member} \hfill {\color{gray} Oct 2017 - Present} \\
\vspace{2mm}
As a member of the Rust compiler team, I participate in the implementation and maintainance of the
Rust compiler. I've contributed and participated in the non-lexical lifetimes (NLL), async/await
and diagnostics working groups, and I co-lead the polymorphisation and meta working groups. Amongst
my most notable contributions, I've implemented the non-exhaustive attribute; polymorphisation,
a code-size and compile-time optimisation which reduces unnecessary monomorphisation; Split DWARF
support; and diagnostic translation.
\vspace{2mm}

\header{Work Experience}
\vspace{1mm}

\textbf{Huawei Technologies Research \& Development UK Ltd.} \hfill {\color{gray} Edinburgh, Scotland} \\
\textit{Senior Software Engineer, Programming Languages Group} \hfill {\color{gray} Aug 2021 - Present} \\
\vspace{2mm}
I primarily contribute to the upstream Rust project, continuing my various ongoing projects,
such as Polymorphisation, Split DWARF and Diagnostic Translation. Concurrently, I assist internal
teams on Rust usage and contributions; give internal talks on Rust and its compiler; and sit on
Huawei's Technical Management Committee for Rust usage in the organisation. \\
\vspace{2mm}
In addition, I contributed to the implementation of automatic differentation and constant evaluation
in an upcoming programming language.
\vspace{2mm}

\textbf{Codeplay Software Ltd.} \hfill {\color{gray} Edinburgh, Scotland} \\
\textit{Senior Software Engineer, Infrastructure} \hfill {\color{gray} Nov 2020 - Aug 2021} \\
\vspace{2mm}
I was the primary maintainer of Codeplay's continuous integration infrastructure and led the effort
to rebuild the core infrastructure with NixOps to improve reproducibility as Product
Owner of Reproducible Infrastructure. \\
\vspace{2mm}
I worked as a compiler engineer on SYCL support for NVIDIA GPUs which was contributed to Intel's
DPC++. I implemented driver support in Clang for the \verb|nvptx64-nvidia-nvcl-sycldevice| target,
target-specific passes in LLVM, builtins in libclc and various bug fixes to LLVM, Clang and the
LLVM-SPIRV translator.
\vspace{2mm}

\textbf{Codeplay Software Ltd.} \hfill {\color{gray} Edinburgh, Scotland} \\
\textit{Software Engineer, Infrastructure} \hfill {\color{gray} Sep 2017 - Nov 2020} \\
\vspace{2mm}

\textbf{Scottish Engineering} \hfill {\color{gray} Glasgow, Scotland} \\
\textit{Software Consultant} \hfill {\color{gray} Sep 2018 - Nov 2018} \\
\vspace{2mm}

\textbf{Codeplay Software Ltd.} \hfill {\color{gray} Edinburgh, Scotland} \\
\textit{Intern Build Engineer} \hfill {\color{gray} May 2017 - Sep 2017} \\
\vspace{2mm}
During a summer, I rebuilt the entirety of Codeplay's continuous integration infrastructure
in my internship - introducing automated re-provisioning of Ubuntu, CentOS and Windows build nodes
and improving the configuration management, vastly reducing the turn-around time of changes
requested by engineering teams and downtime which impacted engineering team productivity. In
addition, I made various improvements to internal tools relied on by engineering teams.
\vspace{2mm}

\textbf{West Dunbartonshire Leisure} \hfill {\color{gray} Alexandria, Scotland} \\
\textit{Software Consultant} \hfill {\color{gray} Apr 2015 - Feb 2017} \\
\vspace{2mm}
Following my work with Polaroid Eyewear on their 10K Race Series website, I worked closely with
West Dunbartonshire Leisure to create a landing page for their 10K race series - a continuation of
Polaroid Eyewear’s series. After consultation with the team at West Dunbartonshire Leisure to
understand their requirements, I built a functional, easy-to-use and easy-to-update website which
requires very little maintenance and launched to great user feedback in February, 2017.
\vspace{2mm}

\textbf{Polaroid Eyewear} \hfill {\color{gray} Dumbarton, Scotland} \\
\textit{Software Consultant} \hfill {\color{gray} Jun 2014 - Jun 2016} \\
\vspace{2mm}
Polaroid Eyewear contracted me to build a shop floor information system - MIDAS - an ASP.NET
system which is the backbone of Polaroid Eyewear's lens manufacturing.
\vspace{2mm}
During the summer of 2015, I built and designed a fully-featured web-based Inventory Manager system
and a suite of desktop applications for tracking the performance of the Warehouse. Both projects
required independence and coordination with other departments and employees. Both projects involved
working closely with users and were released on schedule with brilliant feedback from users. \\
\vspace{2mm}
Following the success of MIDAS, I worked on creating an updated, modern website for Polaroid
Eyewear's 10K Race Series - the e-commerce website allowed runners to enter any of the four races
in the series and be assigned and shipped their race number. After launching the new website to
great feedback, the series saw an increase in runners, charitable donations and saved money on
payment processing costs.
\vspace{2mm}

\header{Education}
\vspace{1mm}

\textbf{University of Glasgow} \hfill {\color{gray} Glasgow, Scotland} \\
\textit{MSci Software Engineering with Work Placement, Honours of the First Class} \hfill {\color{gray} Sep 2015 - Jun 2020} \\
\vspace{2mm}
I graduated with a GPA of 20.0 (out of a maximum 22.0) and completed my Master's Project on
``Polymorphisation'' \footnotemark[1], an code-size optimisation in the Rust compiler to reduce
unnecessary monomorphisation during code generation. In my first year, I was awarded ``Best
Computing Science Student Intending Single Honours'' and in my final year, ``Most Outstanding
Project in MSci SE WP''. \\
\vspace{2mm}
In my third year, I worked in a team tasked with creating a event-sourced financial platform
\footnotemark[2] \footnotemark[3] for Avaloq, a banking software company. For the duration of the
project, I managed and led development on the event bus and the ``superclient''. Both written in
Rust, the event bus is the central server that manages and persists events while ensuring
consistency, correlation and horizontal scaling of microservice clients; the superclient is a
framework for building client applications in Lua with persistence and exposing a REST API. \\
\vspace{2mm}
Additionally, this involved working with the team to design and implement the various solutions that
allowed the system to achieve the desired properties; to streamline and improve our development
processes; and to mentor other team members in fixing bugs and building features when working with
unfamiliar technologies.
\vspace{2mm}

\textbf{Glasgow Caledonian University} \hfill {\color{gray} Glasgow, Scotland} \\
\textit{Nuffield Foundation Placement} \hfill {\color{gray} May 2014 - July 2014} \\
\vspace{2mm}
While on a summer placement at Glasgow Caledonian University, I implemented a colour-based
tracking algorithm from a research paper in C++ with OpenCV \footnotemark[4] \footnotemark[5] which
was capable of full 360 tracking of multiple objects simultaneously including when the object
leaves and re-enters the frame. \\
\vspace{2mm}
Furthermore, I built a tool for non-photorealistic rendering using OpenCV to make an image look
less realistic - in essence, creating a cartoon out of an image. Images were processed in two
distinct stages - extracting the edges from the image and overlaying them on a copy of the
original image that uses a reduced set of colours.
\vspace{2mm}

\textbf{Vale of Leven Academy} \hfill {\color{gray} Alexandria, Scotland} \\
\textit{Secondary Education} \hfill {\color{gray} Aug 2009 - May 2015} \\
\vspace{2mm}

\header{Memberships}
\textbf{British Computer Society} \\
\textit{Professional Membership} \hfill {\color{gray} Jun 2020 - Present} \\
\vspace{2mm}

\textbf{Open Source Initiative} \\
\textit{Individual Membership} \hfill {\color{gray} Feb 2020 - Present} \\
\vspace{2mm}

\header{Published Articles}
\textbf{Inside Rust Blog} \\
\textit{\href{https://blog.rust-lang.org/inside-rust/2022/08/16/diagnostic-effort.html}{Contribute to the diagnostic translation effort!}} \hfill {\color{gray} August 2022} \\
\textbf{Inside Rust Blog} \\
\textit{\href{https://blog.rust-lang.org/inside-rust/2019/10/11/AsyncAwait-Not-Send-Error-Improvements.html}{Improving async-await's ``Future is not Send'' diagnostic}} \hfill {\color{gray} October 2019} \\
\vspace{2mm}

\footnotetext[1]{\href{https://davidtw.co/media/masters_dissertation.pdf}{https://davidtw.co/media/masters{\_}dissertation.pdf}}
\footnotetext[2]{\href{https://davidtw.co/media/autokrator_dissertation.pdf}{https://davidtw.co/media/autokrator{\_}dissertation.pdf}}
\footnotetext[3]{\href{https://davidtw.co/media/autokrator_presentation.pdf}{https://davidtw.co/media/autokrator{\_}presentation.pdf}}
\footnotetext[4]{\href{https://davidtw.co/media/camshift_report.pdf}{https://davidtw.co/media/camshift{\_}report.pdf}}
\footnotetext[5]{\href{https://davidtw.co/media/camshift_poster.pdf}{https://davidtw.co/media/camshift{\_}poster.pdf}}

\end{document}

% vim:foldmethod=marker:foldlevel=0:ts=2:sts=2:sw=2:et:nowrap
